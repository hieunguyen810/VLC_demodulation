\chapter{Kết luận}\label{sec:chapter_4}
\section{Tóm tắt và kết luận chung}

\subsection{Ưu điểm}
\begin{itemize}
\item Xây dựng thành công 3 mô hình có thể phân loại được 2 mức của tín hiệu.
\item So sánh 2 phương pháp phân loại tín hiệu là \ac{pnn} và \ac{grnn} với phương pháp thường được sử dụng là \ac{cnn}
\item Xây dựng mô hình có khả năng cải thiện tín hiệu bị méo dạng.
\item Đề xuất phương pháp xử lý tín hiệu bằng cách tách dữ liệu theo frame.
\item Đề xuất sử dụng xung Lorentz thay thế cho tín hiệu \ac{nrz}.
\end{itemize}

\subsection{Nhược điểm}
\begin{itemize}
\item Tín hiệu chỉ có 2 mức
\item Số tốc độ bit và khoảng cách được khảo sát ít. 
\end{itemize}
\subsection{Nguyên nhân chưa đạt}
\begin{itemize}
\item Chưa giành đủ thời gian nghiên cứu, thí nghiệm.
\item Chưa có đủ kiến thức về generative model.
\end{itemize}

\subsection{Cách khắc phục}
\begin{itemize}
\item Khảo sát thêm các nhiều tốc độ bit ở những khoảng cách khác nhau.
\item Học tập và nghiên cứu sâu về AI và generative model.
\item Thay đổi mạch lái \ac{oled}
\end{itemize}

\section{Hướng phát triển}
\begin{itemize}
\item Xây dựng mô hình bù méo dạng tốt hơn: có thể ứng dụng những phương pháp như \ac{gan}, Adversarial Denoising Autoencoder, ...
\item Kết hợp sức mạch của nhiều mô hình lại với nhau thông qua mô hình Stacking. 
\item Phát triển hệ thống thành MINO
\item Thiết kế mạch lái \ac{oled} phù hợp.
\end{itemize}